\chapter{Typical usage}

{\packageName} includes three main executables (\texttt{evo2sim\_create}, \texttt{evo2sim\_bootstrap} and \texttt{evo2sim\_run}), and a set of executables dedicated to post-treatments, data recovery or tests.

Everything in {\packageName} relies on an ad-hoc file organization where all the data for a simulation is stored: populations in the \texttt{population} directory, environments in \texttt{environment}, phylogenetic and lineage trees in \texttt{tree} and so on. It is not recommended to manually modify these files since this may cause some inconsistency leading to undefined behavior. Besides, most of these files are compressed.

Open a terminal and use the \texttt{cd} command to navigate to {\packageName} directory.
A typical parameters file is provided in {\packageName} package, in folder \texttt{example} (an exhaustive description of the parameters is available in chapter ``Parameters description''). Navigate to this folder using the \texttt{cd} command.
Then follow the steps below for a first usage of the software.

\section{Creating a simulation}

Create a fresh simulation from the parameters file (by default \texttt{parameters.txt}):
\begin{itemize}
        \item[\$] \texttt{../build/bin/evo2sim\_create}
\end{itemize}
Several folders have been created. They mainly contain simulation backups (population, environment, trees, parameters, ...). Additional files and folders have also been created:
\begin{itemize}
        \item \texttt{version.txt}: this file indicates the version of the software. This information is useful to ensure that the code version is compatible with the backup files (\textit{e.g.}, in case of post-treatments).
        \item \texttt{track\_cell.py}: when executed, this python script displays on the fly the internal protein and metabolic concentrations of the cell at position $0\times0$ on the grid. This script is useful to get an idea of internal cell's dynamics (metabolic fluxes, regulation, \ldots).
        \item \texttt{viewer} folder: the viewer is central to the usage of {\packageName} (see chapter ``Simulation viewer''). To access the viewer, open the html page \texttt{viewer/viewer.html} in an internet browser.
\end{itemize}

\section{Generating viable initial conditions with a bootstrap}

Alternatively to the \texttt{evo2sim\_create} executable, use a bootstrap to find a simulation with good initial properties from the parameters file:
\begin{itemize}
        \item[\$] \texttt{../build/bin/evo2sim\_bootstrap}
\end{itemize}
A fresh simulation with an updated parameters file will be automatically created if a suitable seed is found.

\section{Running a simulation}

In {\packageName}, running a simulation necessitates to load it from backup files. Here, we will run a simulation from freshly created backups (see above):
\begin{itemize}
        \item[\$] \texttt{../build/bin/evo2sim\_run -b 0 -t 10000 -g}
\end{itemize}
with \texttt{-b} the date of the backup, here 0 (fresh simulation), \texttt{-t} the simulation time, here 10,000 time-steps. Option \texttt{-g} activates the graphical output (not works if the software has been compiled with the no-graphics option).
At any moment during the simulation, you can take a closer look at the evolution of the system by opening \texttt{viewer/viewer.html} in an internet browser. You can track internal cell's dynamics by executing the script \texttt{track\_cell.py}.

Other main executables are described below in section ``Main executables description''. You can also obtain help by running the executable with the \texttt{-h} option (e.g. \texttt{evo2sim\_create -h})
